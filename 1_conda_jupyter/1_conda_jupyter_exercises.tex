\documentclass[10pt]{article}


\usepackage[dvipsnames]{xcolor}

\usepackage{menukeys}
\usepackage{graphicx}
\usepackage{fancyhdr}
\usepackage{titling}
\usepackage{natbib}
\bibliographystyle{apalike}
\usepackage{datetime}
\usepackage{hyperref}



\usepackage{listings,lstautogobble}

\usepackage[utf8]{inputenc}
\usepackage[english]{babel}
\setlength{\parindent}{8pt}
\usepackage{indentfirst}


\definecolor{codegreen}{rgb}{0,0.6,0}
\definecolor{codegray}{rgb}{0.5,0.5,0.5}
\definecolor{codepurple}{rgb}{0.58,0,0.82}
\definecolor{backcolour}{rgb}{0.95,0.95,0.92}

\lstdefinestyle{mystyle}{
	backgroundcolor=\color{backcolour},   
	commentstyle=\color{codepurple},
	keywordstyle=\color{NavyBlue},
	numberstyle=\tiny\color{codegray},
	stringstyle=\color{codepurple},
	basicstyle=\ttfamily\footnotesize\bfseries,
	breakatwhitespace=false,         
	breaklines=true,                 
	captionpos=t,                    
	keepspaces=true,                 
	numbers=left,                    
	numbersep=5pt,                  
	showspaces=false,                
	showstringspaces=false,
	showtabs=false,                  
	tabsize=2
}

\lstset{style=mystyle}

\settimeformat{ampmtime}

\hypersetup{
	colorlinks=true,
	linkcolor=blue,
	urlcolor=blue,
}


%==================================================
% PAGE SETTING
%==================================================
\usepackage
[
letterpaper,        % US Letter Paper
hmargin=0.5in, % Horizontal Margin
vmargin=1in,     % Vertical Margin
]
{geometry}

\pagestyle{fancy}

%==================================================
% GRAPHICS PATH
%==================================================
\graphicspath{{../figures/}} % this places all graphics in the figures subdirectory


%==================================================
% SPECIFY DOCUMENT TITLE, AUTHOR, AND OPTIONALLY
% SPECIFY A DATE. 
%
% BY DEFAULT, "\today" UPDATES THE DATE
% BASED ON THE CURRENT SYSTEM DATE
%==================================================
\title{Exercises: Conda and Jupyter}
\author{Michael R. Needham}
\date{\today}

\raggedbottom


\begin{document}
	
	%==================================================
	% PLACE AUTHOR NAME IN UPPER LEFT CORNER, DOCUMENT
	% TITLE IN UPPER RIGHT CORNER, AND DATE IN LOWER 
	% RIGHT CORNER, ON EVERY PAGE
	%==================================================
	\lhead{\theauthor}
	\rhead{\thetitle}
	\rfoot{Updated on \thedate \space at  \currenttime}
	
	\begin{center}
		\textbf{\LARGE Exercises for Programming Session 1: }
		
		\textbf{\LARGE Conda Environments and Jupyter Notebooks}		
	\end{center}

	
	\noindent
	Work through these conda and Jupyter exercises here in-person, or at your own pace. For questions, please email Michael (\href{mailto:m.needham@colostate.edu}{m.needham@colostate.edu}). I recommend that you create a new folder to hold this work, as you will download several files over the course of these exercises.
		\section{Install conda on your machine}

\begin{itemize}
\item Follow the instructions in the online \href{https://conda.io/projects/conda/en/latest/user-guide/install/index.html}{conda documentation} to install conda on your local machine.
\item When the installation has completed, ensure it has installed correctly by printing out the version of conda to the terminal:
\end{itemize}		

		
\begin{lstlisting}[language=Python]
	USER$ conda --version
	conda 4.11.0
\end{lstlisting}
	

\section{Create a new conda environment from scratch}
\begin{itemize}
	\item From your terminal window create a new conda environment:
\end{itemize}
\begin{lstlisting}[language=bash]
	USER$ conda create --name env_name
\end{lstlisting}

\begin{itemize}
	\item All environments can be listed on your machine with:
\end{itemize}

\begin{lstlisting}[language=bash]
	USER$ conda env list
	# conda environments:
	#
	env_name                 /path/to/env_name
\end{lstlisting}

\begin{itemize}
	\item You can now activate this environment with:
\end{itemize}


\begin{lstlisting}[language=bash]
	USER$ conda activate env_name
	(env_name) USER$ 
\end{lstlisting}

\section{Install packages in your new environment}

\begin{itemize}
	\item Ensure you are in the correct environment by noticing the string in parenthesis at the beginning of your terminal prompt (see line 2 in the previous code block). Let's install a package:
\end{itemize}

	\begin{lstlisting}[language=bash]
	(env_name) USER$ conda install numpy
\end{lstlisting}

\begin{itemize}
	\item Your terminal will list a lot of output - most of this is not important to you unless you are concerned with particular package versions. When prompted with ``\lstinline|Proceed ([y]/n)?|", proceed by typing `y' for yes and hitting the enter key.
\end{itemize}

\begin{lstlisting}[language=bash]
	Proceed ([y]/n)? y
	...
	Preparing transaction: done
	Verifying transaction: done
	Executing transaction: done
	Retrieving notices: ...working... done
	(env_name) USER$ 
\end{lstlisting}

\begin{itemize}
	\item If you would like, you can list all packages currently installed in the active environment with:
\end{itemize}
\begin{lstlisting}[language=bash]
	(env_name) USER$ conda list
\end{lstlisting}

	
\section{Open a Jupyter Notebook from your environment and run cells}
\begin{itemize}
	\item Open a Jupyter notebook (you may have to \href{https://anaconda.org/anaconda/jupyter}{install Jupyter!}) with:
\end{itemize}

\begin{lstlisting}[language=bash]
	(env_name) USER$ jupyter notebook
\end{lstlisting}

\section{Troubleshoot missing packages}
\begin{itemize}
	\item Download the file \href{https://github.com/m-needham/first_year_programming_sequence/blob/main/1_conda_jupyter/squares.ipynb}{ \lstinline|squares.ipynb|} from my GitHub. First, select ``raw" to view the raw text, then save\footnote{On mac, this is easily done with \cmd S. Likely is something similar on PC (CTRL-S?)} this text to a local folder on your machine.
	
	\item Open \lstinline|squares.ipynb| in Jupyter Notebooks and run the entire notebook. If you are running from your newly created conda environment, you should get the error  \lstinline[language=Python]|ModuleNotFoundError: No module named `matplotlib'|  after cell 4
	
	\item Fix this error by installing matplotlib with conda, restarting the kernel, and running all cells 
	
\end{itemize}

\begin{lstlisting}[language=bash]
	(env_name) USER$ conda install -c conda-forge matplotlib
\end{lstlisting}

\section{Create a new conda environment from an existing file}

\begin{itemize}
	\item Download the files \href{https://github.com/m-needham/first_year_programming_sequence/blob/main/1_conda_jupyter/environment.yml}{\lstinline|environment.yml|}, \href{https://github.com/m-needham/first_year_programming_sequence/blob/main/1_conda_jupyter/fibonacci.csv}{\lstinline|fibonacci.csv|}, and \href{https://github.com/m-needham/first_year_programming_sequence/blob/main/1_conda_jupyter/fibonacci.ipynb}{\lstinline|fibonacci.ipynb|} from the GitHub, as before. In your terminal, navigate to the folder holding \lstinline|environment.yml| and create a new conda environment using its contents: 
\end{itemize}


\begin{lstlisting}[language=bash]
	(env_name) USER$ conda env create -f environment.yml 
\end{lstlisting}

\begin{itemize}
	\item This will create a new conda environment \lstinline|shared_env| with the packages necessary\footnote{The only difference is that this one includes \href{https://pandas.pydata.org/}{\lstinline|pandas|} to read a \lstinline|csv| file more easily} to run \lstinline|fibonacci.ipynb|
	\item Open \lstinline|Jupyter Notebooks| and ensure you can now run \lstinline|fibonacci.ipynb|
\end{itemize}
	
	%\pagebreak
	%\bibliography{bibfile}
\end{document}
